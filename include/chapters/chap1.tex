\chapter{Nozioni Fondamentali}

\section{Problemi di Ottimizzazione}
Un problema di ottimizzazione $\mathcal{P}$ può essere espresso con la
formulazione generale
\begin{equation}\label{eq:opt_prob}
    \mathcal{P} \colon
    \begin{cases}
        \min \text{(or $\max$)} & f(\vec{x}) \\
                                & \mathcal{S} \\
                                & \vec{x} \in \mathcal{D}
    \end{cases}
\end{equation}
dove $f(\vec{x})$ è una funzione a valori reali con $\vec{x} = [x_{1},
\ldots, x_{n}]^T \in \mathcal{D} = D_{1} \times \cdots \times D_{n}$ tale
che $x_{j} \in D_{j},\, \forall j\colon 1 \le j \le n,\, n \in \mathbb{N}$
e $\mathcal{S}$ è un insieme finito di vincoli. Formalmente, un vincolo $c
\in \mathcal{S}$ è una funzione che coinvolge un sottoinsieme di variabili
e che può assumere i valori vero o falso, che si riferiscono alla
condizione di vincolo soddisfatto o violato, rispettivamente.

\begin{definition}
Dato un problema di ottimizzazione $\mathcal{P}$, ogni $\vec{x} \in
\mathcal{D}$ è detto \textit{soluzione} di $\mathcal{P}$, con $\mathcal{D}$
il dominio delle variabili del problema. Una soluzione $\vec{x} \in
\mathcal{D}$ che soddisfi tutti i vincoli in $\mathcal{S}$ è detta
\textit{ammissibile} per $\mathcal{P}$.
\end{definition}
Per riferirci all'insieme di tutte le soluzioni ammissibili per un problema
$\mathcal{P}$ utilizziamo la notazione $F(\mathcal{P})$. La
caratterizzazione del dominio $\mathcal{D}$ fornisce un'immediata
classificazione dei problemi di ottimizzazione.
\begin{itemize}
    \item Se il dominio $\mathcal{D}$ è un insieme discreto, il problema è
        detto di \textit{ottimizzazione discreta}.
    \item Se il dominio $\mathcal{D}$ è un insieme continuo, il problema è
        detto di \textit{ottimizzazione continua}.
\end{itemize}
Nel caso particolare di un dominio $\mathcal{D}$ che sia un insieme
discreto e finito, si parla di \textit{ottimizzazione combinatoria}.
\begin{definition}
Sia $\mathcal{P}$ un problema di minimizzazione. Una soluzione ammissibile
$\vec{x^{\star}} \in F(\mathcal{P})$ si dice \textit{ottima} per
$\mathcal{P}$ se vale che
\[
    f(\vec{x^{\star}}) \le f(\vec{x}) \quad
    \forall \vec{x} \in F(\mathcal{P}).
\]
\end{definition}
Una definzione speculare vale per problemi di massimizzazione. In generale,
ricercare il massimo di una funzione $f(\vec{x})$ è equivalente a ricercare
il minimo per la funzione ${-f(\vec{x})}$. Per questo motivo, nel seguito
potremo riferirci a problemi di minimo senza perdere di generalità. Tutte
le considerazioni valgono, con le opportune modifiche, anche per i problemi
di massimo.

Un problema di ottimizzazione $\mathcal{P}$ è \textit{impossibile}
(infeasible) quando non ha soluzioni ammissibili, ossia quando
$F(\mathcal{P}) = \varnothing$. Al contrario, si dice che $\mathcal{P}$
ammette ottimo finito quando esiste una soluzione $\vec{x^{\star}} \in
F(\mathcal{P})$ ottima per $\mathcal{P}$. In questo caso,
$f(\vec{x^{\star}})$ è detto valore ottimo per $\mathcal{P}$. Un problema di
ottimizzazione $\mathcal{P}$ si dice \textit{illimitato} (unbounded),
quando non esiste alcun limite inferiore a $f(\vec{x})$, per $\vec{x} \in
F(\mathcal{P})$.

Infine, un problema di ottimizzazione si dice risolto quando viene trovata una
soluzione ottima, accompagnata dal certificato di ottimalità, oppure
quando si riesce a dimostrare che il problema è impossible o illimitato.

\subsection{Restrizioni e Rilassamenti}
\begin{definition}[Restrizione]
Dato un problema di ottimizzazione $\mathcal{P}$, si definisce
\textit{restrizione} di $\mathcal{P}$ un problema di ottimizzazione
$\mathcal{P}'$ ottenuto a partire da $\mathcal{P}$ aggiungendo vincoli.
Formalmente risulta che $F(\mathcal{P}') \subseteq F(\mathcal{P})$.
\end{definition}
Una conseguenza immediata di questa definizione è la seguente. Se una
soluzione $\bar{\vec{x}}$ è ammissibile per $\mathcal{P}'$, ossia
$\bar{\vec{x}} \in F(\mathcal{P}')$, allora è anche ammissibile per
$\mathcal{P}$, cioè $\bar{\vec{x}} \in F(\mathcal{P})$. Se poi
$\vec{x^{\star}} \in F(\mathcal{P})$ è ottima per $\mathcal{P}$, allora
vale $f(\vec{x^{\star}}) \le f(\bar{\vec{x}})$. Ed allora, il valore
associato alla generica soluzione ammissibile $\bar{\vec{x}}$ di
$\mathcal{P}'$ fornisce un limite superiore (\textit{upper bound}) al
valore ottimo di $\mathcal{P}$. Inoltre, dalla definzione segue
che se $\mathcal{P}$ è impossibile, ossia $F(\mathcal{P}) = \varnothing$,
allora anche ogni sua restrizione $\mathcal{P}'$ è impossibile, cioè
$F(\mathcal{P}') = \varnothing$. Non vale il viceversa.
\begin{definition}[Rilassamento]
Dato un problema di ottimizzazione $\mathcal{P}$, si definisce
\textit{rilassamento} di $\mathcal{P}$ un problema di ottimizzazione
$\mathcal{R}$ ottenuto a partire da $\mathcal{P}$ eliminando alcuni vincoli
e/o sostituendo la funzione obiettivo $f(\vec{x})$ di $\mathcal{P}$ con una
sua approssimazione inferiore $g(\vec{x}).$ Formalmente, $\mathcal{R}$ è un
rilassamento di $\mathcal{P}$ se $F(\mathcal{P}) \subseteq F(\mathcal{R})$
e $g(\vec{x}) \le f(\vec{x}),\, \forall \vec{x} \in F(\mathcal{P})$.
\end{definition}
Da questa definizione segue immediatamente che se $\mathcal{R}$ è
impossibile, ossia $F(\mathcal{R}) = \varnothing$, allora anche
$\mathcal{P}$ è impossibile, cioè $F(\mathcal{P}) = \varnothing$. Non vale
il viceversa. Inoltre, è facile verificare che se $\vec{x^{\star}} \in
F(\mathcal{R})$ è ottima per $\mathcal{R}$, allora $g(\vec{x^{\star}})$ è
un limite inferiore (\textit{lower bound}) al valore ottimo di
$\mathcal{P}$. Se invece vale che $\vec{x^{\star}} \in F(\mathcal{R})$ è
ottima per $\mathcal{R}$ e contemporaneamente $\vec{x^{\star}} \in
F(\mathcal{P})$, ossia $\vec{x^{\star}}$ è ammissibile per $\mathcal{P}$,
con $g(\vec{x^{\star}}) = f(\vec{x^{\star}})$, si dimostra facilmente che
$\vec{x^{\star}}$ è soluzione ottima di $\mathcal{P}$.

\subsection{Ricerca}
\begin{definition}
    Sia $\mathcal{P}$ un problema di ottimizzazione. Si definisce
    \textit{ricerca} il \mbox{processo} che consiste nel risolvere una
    sequenza finita $\mathcal{P}_1, \ldots, \mathcal{P}_m$ di restrizioni
    di $\mathcal{P}$.
\end{definition}
L'intuizione alla base della ricerca è la seguente. Aggiungendo dei
vincoli, nel processo di creazione delle restrizioni, si riduce lo
spazio delle soluzioni ammissibili con l'obiettivo semplificare la
risoluzione del problema.

Una ricerca è detta \textit{esaustiva} se la sequenza delle restrizioni cui
fa riferimento copre tutto lo spazio $F(\mathcal{P})$ delle soluzioni
ammissibili di $\mathcal{P}$. Formalmente, deve valere che
\begin{equation}
    \bigcup_{i=1}^{m}F(\mathcal{P}_i) = F(\mathcal{P}).
\end{equation}
Una ricerca esaustiva fornisce quindi un modo per risolvere $\mathcal{P}$:
è sufficiente risolvere tutte le restrizioni $\mathcal{P}_i$, per $1 \le i
\le m$, e considerare la soluzione migliore.  Inevitabilmente, tale
soluzione sarà quella ottima per $\mathcal{P}$. Una ricerca non esaustiva è
invece detta euristica.

Il concetto di ricerca è del tutto generale e si applica differentemente
a seconda del contesto cui si fa riferimento. Una delle forme di
ricerca esaustiva è chiamata \textit{generate-and-test}. Come suggerisce il
nome, l'idea è quella di generare esplicitamente tutte le soluzioni
$\vec{x} \in \mathcal{D}$ e verificare quali tra queste soddisfano i
vincoli del problema che stiamo considerando, per poi scegliere la
migliore. Questo equivale a creare delle restrizioni aggiungendo in
ciascuna un vincolo relativamente al valore assunto da $\vec{x} \in
\mathcal{D}$. Questa forma di ricerca è molto semplice ma non molto
efficiente e per questo motivo è applicabile ad una classe molto ristretta
di problemi di ottimizzazione.

Una forma di ricerca migliore, che costituisce il punto di partenza per
tutti gli algoritmi enumerativi utilizzati nei contesti reali, è la ricerca
ad albero (\textit{tree-search}). L'idea è quella di dividere
ricorsivamente lo spazio di ricerca di $\mathcal{P}$, fino ad ottenere
restrizioni sufficientemente facili da risolvere. Uno schema di ricerca di
questo tipo è spesso combinato con il concetto di rilassamento. L'obiettivo
è quello di semplificare la soluzione di una restrizione utilizzando un suo
rilassamento.

\section{Programmazione Lineare}

