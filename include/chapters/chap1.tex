\chapter{Nozioni Fondamentali}
L'obiettivo di questo capitolo è quello di presentare i concetti di base
che vengono utilizzati nel seguito di questo lavoro e, più in generale,
nell'ambito dell'ottimizzazione matematica.

\section{Introduzione}
Ottimizzare significa risolvere un problema trovando la soluzione migliore
nell'insieme delle soluzioni che abbiamo a disposizione. Il desiderio di
scegliere la soluzione migliore è così comune che il concetto di
ottimizzazione è parte integrante di ogni ambito che richieda di risolvere
problemi decisionali. Nella pratica, risolvere un problema decisionale
significa assegnare valori alle variabili che lo caratterizzano, con
la finalità di ottimizzare una grandezza, rispettando un insieme di
vincoli che limitano le scelte dei valori per queste variabili. La
grandezza da ottimizzare e i vincoli possono essere spesso
rappresentati come funzioni delle variabili coinvolte e questo ci permette
di definire e utilizzare un modello matematico per descrivere il problema che stiamo
considerando. La formulazione di un modello dovrebbe essere
sufficientemente complessa da rappresentare accuratamente il problema cui
si riferisce e, allo stesso tempo, abbastanza semplice da renderlo
trattabile con gli strumenti risolutivi disponibili.

La ricerca operativa è un ramo della matematica applicata che impiega
metodi e modelli matematici per risolvere problemi decisionali. Di fronte
ad un problema del mondo reale, il primo passo è quello di trovare un modello
matematico che sia in grado di descriverlo. Successivamente si procede con
l'applicazione dei metodi di ottimizzazione per trovare la soluzione
migliore. Infine, la soluzione trovata va interpretata e adattata al
problema reale. Questo processo può anche essere iterato, con l'obiettivo di
raffinare il modello di riferimento e ottenere
soluzioni sempre più accurate.

Non tutte le classi di problemi di ottimizzazione si possono risolvere
all'ottimo in un tempo ragionevole e questo ha portato allo sviluppo di
algoritmi euristici.

\section{Problemi di Ottimizzazione}
Un problema di ottimizzazione $\mathcal{P}$ può essere definito con la
formulazione generale

\begin{equation}
    \mathcal{P}\colon
    \begin{cases}
        \text{$\min$ (or $\max$)} & f(\vec{x}) \\
                                  & \mathcal{S} \\
                                  & \vec{x} \in \mathcal{D} \\
    \end{cases}
\end{equation}
dove $f(\vec{x})$ è una funzione a valori reali nelle variabili $\vec{x} =
[x_{1}\, \ldots \, x_{n}]^T \in \mathcal{D} = D_{1} \times
\cdots \times D_{n}$ con $x_{j} \in D_{j}\,\,\,\forall j\colon 1 \le j
\le n,\, n \in \mathbb{N}$ e $\mathcal{S}$ è un insieme finito di vincoli.
Formalmente, un vincolo $c \in \mathcal{S}$ è una funzione che coinvolge un
sottoinsieme delle variabili del problema e che può assumere i valori vero
o falso, corrispondenti alla condizione di vincolo soddisfatto o violato,
rispettivamente.

Il massimo valore di una funzione $f(\vec{x})$ è il minimo valore della
funzione $-f(\vec{x})$, a meno del segno. Per questo motivo possiamo
limitarci a studiare i problemi di minimo senza perdere di generalità.
Tutte le considerazioni saranno valide, con opportune modifiche, anche per
i problemi di massimo.

\begin{definition}
Dato un problema di ottimizzazione $\mathcal{P}$, ogni $\vec{x} \in
\mathcal{D}$ si dice soluzione di $\mathcal{P}$. Una soluzione di
$\mathcal{P}$ che soddisfi tutti i vincoli in $\mathcal{S}$ si dice
ammissibile per $\mathcal{P}$.
\end{definition}
Per riferirci all'insieme di tutte le soluzioni ammissibili per un problema
di ottimizzazione $\mathcal{P}$, utilizzeremo la notazione
$F(\mathcal{P})$.

La caratterizzazione del dominio $\mathcal{D}$ fornisce una
classificazione immediata dei problemi di ottimizzazione.

\begin{itemize}
    \item Se il dominio $\mathcal{D}$ è un insieme discreto, il problema è
        detto di \textit{ottimizzazione discreta}.
    \item Se il dominio $\mathcal{D}$ è un insieme continuo, il problema è
        detto di \textit{ottimizzazione continua}.
\end{itemize}
Inoltre, nel caso particolare di un dominio $\mathcal{D}$ che un insieme
discreto e finito, si parla di \textit{ottimizzazione combinatoria}.

\begin{definition}
Sia $\mathcal{P}$ un problema di ottimizzazione. Una soluzione ammissibile
$\vec{x^{\star}} \in F(\mathcal{P})$ si dice ottima per $\mathcal{P}$ se
\[
    f(\vec{x^{\star}}) \le f(\vec{x}) \quad \forall \vec{x} \in
    F(\mathcal{P}).
\]
\end{definition}
Un problema di ottimizzazione $\mathcal{P}$ si dice impossibile
(\textit{infeasible}) quando non ammette soluzioni ammissibili, ossia
quando $F(\mathcal{P}) = \varnothing$. Diciamo invece che $\mathcal{P}$ è
illimitato (\textit{unbounded}) quando non esiste alcun limite inferiore a
$f(\vec{x})$, per $\vec{x} \in F(\mathcal{P})$. Se esiste una soluzione
$\vec{x^{\star}} \in \mathcal{P}$ ottima per $\mathcal{P}$, allora diciamo
che $\mathcal{P}$ ammette ottimo finito. La funzione $f(\vec{x})$ è
generalmente chiamata funzione obiettivo e il valore $f(\vec{\bar{x}})$ è
detto costo associato alla soluzione $\vec{\bar{x}} \in F(\mathcal{P})$.

Un problema di ottimizzazione si dice risolto quando si trova una soluzione
ottima, accompagnata dal certificato di ottimalità, oppure quando si riesce
a dimostrare che il problema è impossibile o illimitato.

\section{Programmazione Lineare}
\section{Programmazione Lineare Intera}

